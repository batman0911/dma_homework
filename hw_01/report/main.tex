%%%%%%%%%%%%%%%%%%%%%%%%%%%%%%%%%%%%%%%%%%%%%%%%
%% Intro to LaTeX and Template for Homework Assignments
%% Quantitative Methods in Political Science
%% University of Mannheim
%% Fall 2019
%%%%%%%%%%%%%%%%%%%%%%%%%%%%%%%%%%%%%%%%%%%%%%%%

% created by Marcel Neunhoeffer & Sebastian Sternberg


% This template and tutorial will help you to write up your homework. It will also help you to use Latex for other assignments than this course's homework.

%%%%%%%%%%%%%%%%%%%%%%%%%%%%%%%%%%%%%%%%%%%%%%%%
% Before we get started
%%%%%%%%%%%%%%%%%%%%%%%%%%%%%%%%%%%%%%%%%%%%%%%%

% Make an account on overleaf.com and get started. No need to install anything.

%%%%%%%%%%%%%%%%%%%%%%%%%%%%%%%%%%%%%%%%%%%%%%%%
% Or if you want it the nerdy way...
% INSTALL LATEX: Before we can get started you need to install LaTeX on your computer.
				% Windows: http://miktex.org/download
				% Mac:         http://www.tug.org/mactex/mactex-download.html	
				% There a many more different LaTeX editors out there for both operating systems. I use TeXworks because it looks the same on Windows and Mac.
				

% SAVE THE FILE: The first thing you need to do is to save your LaTeX file in a directory as a .tex file. You will not be able to do anything else unless your file is saved. I suggest to save the .tex file in the same folder with your .R script and where you will save your plots from R to. Let's call this file template_homework1.tex and save it in your Week 1 folder.


% COMPILE THE FILE: After setting up your file, using your LaTeX editor (texmaker, texshop), you can compile your document using PDFLaTeX.
	% Compiling your file tells LaTeX to take the code you have written and create a pdf file
	% After compiling your file, in your directory will appear four new files, including a .pdf file. This is your output document.
	% It is good to compile your file regularly so that you can see how your code is translating into your document.
	
	
% ERRORS: If you get an error message, something is wrong in your code. Fix errors before they pile up!
	% As with error messages in R, google the exact error message if you have a question!
%%%%%%%%%%%%%%%%%%%%%%%%%%%%%%%%%%%%%%%%%%%%%%%%


% Now again for everyone...

% COMMANDS: 
	% To do anything in LaTeX, you must use commands
	% Commands tell LaTeX when to start your document, how you want your document to look, and how to format your document
	% Commands ALWAYS begin with a backslash \

% Everything following the % sign is a comment and will not be used by Latex to compile your document.
% This is very similar to # comments in R.

% Every .tex file usually consists of four parts.
% 1. Document Class
% 2. Packages
% 3. Header
% 4. Your Document

%%%%%%%%%%%%%%%%%%%%%%%%%%%%%%%%%%%%%%%%%%%%%%%%
% 1. Document Class
%%%%%%%%%%%%%%%%%%%%%%%%%%%%%%%%%%%%%%%%%%%%%%%%
 
 % The first command you will always have will declare your document class. This tells LaTeX what type of document you are creating (article, presentation, poster, etc). 
% \documentclass is the command
% in {} you specify the type of document
% in [] you define additional parameters
 
\documentclass[a4paper,12pt]{article} % This defines the style of your paper

% We usually use the article type. The additional parameters are the format of the paper you want to print it on and the standard font size. For us this is a4paper and 12pt.

%%%%%%%%%%%%%%%%%%%%%%%%%%%%%%%%%%%%%%%%%%%%%%%%
% 2. Packages
%%%%%%%%%%%%%%%%%%%%%%%%%%%%%%%%%%%%%%%%%%%%%%%%

% Packages are libraries of commands that LaTeX can call when compiling the document. With the specialized commands you can customize the formatting of your document.
% If the packages we call are not installed yet, TeXworks will ask you to install the necessary packages while compiling.

% First, we usually want to set the margins of our document. For this we use the package geometry. We call the package with the \usepackage command. The package goes in the {}, the parameters again go into the [].
\usepackage[top = 2.5cm, bottom = 2.5cm, left = 2.5cm, right = 2.5cm]{geometry} 

% Unfortunately, LaTeX has a hard time interpreting German Umlaute. The following two lines and packages should help. If it doesn't work for you please let me know.
\usepackage[T1]{fontenc}
\usepackage[utf8]{inputenc}

% The following two packages - multirow and booktabs - are needed to create nice looking tables.
\usepackage{multirow} % Multirow is for tables with multiple rows within one cell.
\usepackage{booktabs} % For even nicer tables.

% As we usually want to include some plots (.pdf files) we need a package for that.
\usepackage{graphicx} 

% The default setting of LaTeX is to indent new paragraphs. This is useful for articles. But not really nice for homework problem sets. The following command sets the indent to 0.
\usepackage{setspace}
\setlength{\parindent}{0in}

% Package to place figures where you want them.
\usepackage{float}

% The fancyhdr package let's us create nice headers.
\usepackage{fancyhdr}

\usepackage[utf8]{vietnam}

\usepackage{tikz}

\usepackage{graphicx}

\usepackage{listings}

% Import custom code block
\input{config/code_block.tex}


%%%%%%%%%%%%%%%%%%%%%%%%%%%%%%%%%%%%%%%%%%%%%%%%
% 3. Header (and Footer)
%%%%%%%%%%%%%%%%%%%%%%%%%%%%%%%%%%%%%%%%%%%%%%%%

% To make our document nice we want a header and number the pages in the footer.

\pagestyle{fancy} % With this command we can customize the header style.

\fancyhf{} % This makes sure we do not have other information in our header or footer.

\lhead{\footnotesize QM 2019: Homework 1}% \lhead puts text in the top left corner. \footnotesize sets our font to a smaller size.

%\rhead works just like \lhead (you can also use \chead)
\rhead{\footnotesize Lastname 1, Lastname 2 (\& Lastname 3)} %<---- Fill in your lastnames.

% Similar commands work for the footer (\lfoot, \cfoot and \rfoot).
% We want to put our page number in the center.
\cfoot{\footnotesize \thepage} 


%%%%%%%%%%%%%%%%%%%%%%%%%%%%%%%%%%%%%%%%%%%%%%%%
% 4. Your document
%%%%%%%%%%%%%%%%%%%%%%%%%%%%%%%%%%%%%%%%%%%%%%%%

% Now, you need to tell LaTeX where your document starts. We do this with the \begin{document} command.
% Like brackets every \begin{} command needs a corresponding \end{} command. We come back to this later.

\begin{document}


%%%%%%%%%%%%%%%%%%%%%%%%%%%%%%%%%%%%%%%%%%%%%%%%
%%%%%%%%%%%%%%%%%%%%%%%%%%%%%%%%%%%%%%%%%%%%%%%%

%%%%%%%%%%%%%%%%%%%%%%%%%%%%%%%%%%%%%%%%%%%%%%%%
% Title section of the document
%%%%%%%%%%%%%%%%%%%%%%%%%%%%%%%%%%%%%%%%%%%%%%%%

% For the title section we want to reproduce the title section of the Problem Set and add your names.

\thispagestyle{empty} % This command disables the header on the first page. 

\begin{tabular}{p{15.5cm}} % This is a simple tabular environment to align your text nicely 
{\large \bf Toán rời rạc và thuật toán} \\
Đại học Khoa học Tự nhiên \\
Khoa Toán - Cơ - Tin học \\ 
Khoa học dữ liệu K4 \\
Tháng 8 năm 2022  \\ 
\hline % \hline produces horizontal lines.
\\
\end{tabular} % Our tabular environment ends here.

\vspace*{0.3cm} % Now we want to add some vertical space in between the line and our title.

\begin{center} % Everything within the center environment is centered.
	{\Large \bf Bài tập số 1} % <---- Don't forget to put in the right number
	\vspace{2mm}
	
        % YOUR NAMES GO HERE
	{\bf Nguyễn Mạnh Linh, Nguyễn Thị Đông, Triệu Hồng Thúy} % <---- Fill in your names here!
		
\end{center}  

\vspace{0.4cm}

%%%%%%%%%%%%%%%%%%%%%%%%%%%%%%%%%%%%%%%%%%%%%%%%
%%%%%%%%%%%%%%%%%%%%%%%%%%%%%%%%%%%%%%%%%%%%%%%%

% Up until this point you only have to make minor changes for every week (Number of the homework). Your write up essentially starts here.

\section{Bài 1}
\subsection{Những khái niệm cơ bản về đồ thị}
\subsubsection{Khái niệm đồ thị}
Một đồ thị là một cấu trúc rời rạc gồm tập hợp các đỉnh và các cạnh nối giữa các đỉnh đó. Có thể mô tả đồ thị theo cách hình thức như sau:\\
\centerline{\textbf{G = (V,E)}}
tức là đồ thị \textbf{G} có tập các đỉnh là \textbf{V}, tập các cạnh là \textbf{E}. Có thể hiểu \textbf{E} là tập hợp các cặp (u, v) với u và v là hai đỉnh thuộc V.\\
Tập các đỉnh V của đồ thị G có thể là vô hạn. Một đồ thị có tập đỉnh vô hạn hoặc vô số cạnh được gọi là đồ thị vô hạn, và khi so sánh, đồ thị có tập đỉnh hữu hạn và tập hợp cạnh hữu hạn được gọi là đồ thị hữu hạn
\begin{figure}[H] % places figure environment here   
    \centering % Centers Graphic
    \includegraphics[width=0.6\textwidth]{assets/computer_network.png} 
    \caption{Mạng máy tính} % Creates caption underneath graph
    \label{fig:gr_1.1.1}
\end{figure}
Bây giờ, giả sử rằng một mạng được tạo thành từ các trung tâm dữ liệu và các liên kết giao tiếp giữa các máy tính. Chúng ta có thể biểu diễn vị trí của mỗi trung tâm dữ liệu bằng một điểm và mỗi liên kết truyền thông liên kết bằng một đoạn thẳng, như thể hiện trong Hình 1.\\
\subsubsection{Phân loại đồ thị}
Có thể phân loại đồ thị G theo tính chất các tập cạnh như sau:\\
\begin{figure}[H] % places figure environment here   
    \centering % Centers Graphic
    \includegraphics[width=0.8\textwidth]{assets/phanloaidothi.png} 
    \caption{Phân loại đồ thị} % Creates caption underneath graph
    \label{fig:gr_1.2.2}
\end{figure}
\begin{enumerate}
\item{\textbf{Đơn đồ thị}} \\
    G được gọi là đơn đồ thị nếu như giữa hai đỉnh (u, v) của V có nhiều nhất một cạnh trong E nối từ u tới v.\\
    Như Hình 1 là một đơn đồ thị. Mạng máy tính này có thể được mô hình hóa bằng cách sử dụng một đồ thị trong đó các đỉnh của đồ thị đại diện cho các trung tâm dữ liệu và các cạnh đại diện cho các liên kết truyền thông. Lưu ý rằng mỗi cạnh của đồ thị đại diện cho mạng máy tính này kết nối hai đỉnh khác nhau. Có nghĩa là, không có cạnh nào kết nối một đỉnh với chính nó. Hơn nữa, không có hai cạnh khác nhau nối cùng một cặp đỉnh.
\item{\textbf{Đa đồ thị}} \\
    G được gọi là đa đồ thị nếu như giữa hai đỉnh (u, v) của V có thể có nhiều hơn 1 cạnh nối trong E nối từ u tới v. Hiển nhiên đơn đồ thị cũng là đa đồ thị.
    \begin{figure}[H] % places figure environment here   
        \centering % Centers Graphic
        \includegraphics[width=0.6\textwidth]{assets/da_dothi.png} 
        \caption{Đa đồ thị} % Creates caption underneath graph
        \label{fig:gr_1.1.3}
    \end{figure}
    Một mạng máy tính có thể chứa nhiều liên kết giữa các trung tâm dữ liệu, như trong Hình 3. Để mô hình hóa các mạng như vậy, chúng ta cần các đồ thị có nhiều hơn một cạnh nối cùng một cặp đỉnh. Các đồ thị có thể có nhiều cạnh nối các đỉnh giống nhau được gọi là đồ thị đa phương. Khi có m cạnh khác nhau liên kết với cùng một cặp đỉnh không có thứ tự {u, v}, ta cũng nói rằng {u, v} là một cạnh bội số m. Tức là, chúng ta có thể coi tập hợp các cạnh này là m bản sao khác nhau của một cạnh {u, v}.
\item{\textbf{Đồ thị vô hướng}} \\
    G được gọi là đồ thị vô hướng (undirected graph) nếu như các cạnh trong E là không định hướng, tức là cạnh (u, v) là cạnh hai chiều.\\
\item{\textbf{Đồ thị có hướng}} \\
    G được gọi là đồ thị có hướng (directed graph) nếu như các cạnh trong E là có định hướng.\\
    Khi chúng ta mô tả một đồ thị có hướng bằng một bản vẽ đường thẳng, chúng ta sử dụng một mũi tên chỉ từ u đến v để chỉ ra hướng của một cạnh bắt đầu tại u và kết thúc tại v. Một biểu đồ có hướng có thể chứa các vòng lặp và nó có thể chứa nhiều cạnh có hướng bắt đầu và kết thúc ở cùng một đỉnh. Một đồ thị có hướng cũng có thể chứa các cạnh có hướng nối các đỉnh u và v theo cả hai hướng; nghĩa là, khi một đồ thị chứa một cạnh từ u đến v, nó cũng có thể chứa một hoặc nhiều cạnh từ v đến u.\\
    Khi một đồ thị có hướng không có vòng lặp và không có nhiều cạnh có hướng, nó được gọi là \textbf{đồ thị có hướng đơn giản}. Bởi vì một đồ thị có hướng đơn giản có nhiều nhất một cạnh liên kết với mỗi cặp đỉnh có thứ tự (u, v), chúng ta gọi (u, v) là một cạnh nếu có một cạnh liên kết với nó trong đồ thị.\\
    Các đồ thị có hướng có thể có nhiều cạnh có hướng từ một đỉnh đến một đỉnh thứ hai (có thể giống nhau) được sử dụng để mô hình hóa các mạng như vậy. Những đồ thị như vậy là \textbf{đồ thị đa hướng}. Khi có m cạnh có hướng, mỗi cạnh liên kết với một cặp đỉnh có thứ tự (u, v), chúng ta nói rằng (u, v) là một cạnh của m.\\
    \begin{figure}[H] % places figure environment here   
        \centering % Centers Graphic
        \includegraphics[width=0.6\textwidth]{assets/dothi_dahuong.png} 
        \caption{Đồ thị đa hướng} % Creates caption underneath graph
        \label{fig:gr_1.1.4}
    \end{figure}
\end{enumerate}
\subsubsection{Đường đi và chu trình}
Gọi n là một số nguyên không âm và G là một đồ thị vô hướng. 
Đường đi có độ dài n từ u đến v trong G là dãy gồm n cạnh $e_1, ..., e_n$ của G mà trong đó tồn tại dãy $x_0 = u, x_1, ..., x_{n-1}, x_n = v$ thuộc các đỉnh sao cho $e_i$, với i = 1, ..., n, các điểm cuối $x_{i-1}$ và $x_i$. 
Khi đồ thị đơn giản, ta biểu thị đường đi này bằng dãy đỉnh $x_0, x_1, ..., x_n$ (vì liệt kê các đỉnh này xác định duy nhất đường đi).\\
Đường đi là \textbf{một chu trình} nếu nó bắt đầu và kết thúc tại cùng một đỉnh, nghĩa là, nếu u = v, và có độ dài lớn hơn 0. Đường đi hoặc chu trình được cho là đi qua các đỉnh $x_1, x_2, ..., x_{n-1}$ hoặc đi qua các cạnh $e_1, ..., e_n$. Một đường đi hoặc chu trình sẽ đơn giản nếu nó không chứa cùng một cạnh nhiều hơn một lần.\\
Đường đi được gọi là đường đi đơn giản (simple path) nếu tất cả các đỉnh trên đường đi đó đều phân biệt. Đường đi được gọi là đường đi đơn nếu như không có cạnh nào trên đường đi đó đi qua hơn một lần.\\
Chu trình gọi là chu trình đơn giản (simple circuit) nếu như {$x_1, x_2,..., x_k$}v đôi một khác nhau. Chu trình mà trong đó không có cạnh nào đi qua hơn một lần được gọi là chu trình đơn.\\
\textit{Ví dụ:}\\
\begin{figure}[H] % places figure environment here   
    \centering % Centers Graphic
    \includegraphics[width=0.5\textwidth]{assets/duongdi_dongian.png} 
    \caption{Đơn đồ thị} % Creates caption underneath graph
\end{figure}
Trong đồ thị trên, a, d, c, f, e là một đường đi đơn giản có độ dài 4, bởi vì {a, d}, {d, c}, {c, f} và {f, e } là tất cả các cạnh. Tuy nhiên, d, e, c, a không phải là đường đi, vì {e, c} không phải là cạnh. Lưu ý rằng b, c, f, e, b là một chu trình có độ dài 4 vì {b, c}, {c, f}, {f, e} và {e, b} là các cạnh và đường đi này bắt đầu và kết thúc ở b. Đường đi a, b, e, d, a, b có độ dài 5 không phải là đường đi đơn vì nó chứa cạnh {a, b} hai lần.

\subsection{Cấu trúc dữ liệu biểu diễn đồ thị}
\subsubsection{Thiết kế phần mềm}
\textit{Đồ thị các thành phần phụ thuộc (Module Dependency Graphs)} \\
Một trong những nhiệm vụ quan trọng trong thiết kế phần mềm đó là chia chương trình
thành nhiều thành phần hoặc module khác nhau để tiện cho việc phát triển mà mở rộng
cũng như bảo trì sau này. Việc hiểu được sự tương tác giữa các modules trong một 
chương trình tương tác với nhau như thế nào là cực kì quan trọng không chỉ trong việc 
thiết kế chương trình mà còn trong việc kiểm thử và bảo trì nữa. Một đồ thị các thành 
phần phụ thuộc giúp ích rất nhiều trong việc này. Trong đồ thị các thành phần phụ thuộc 
(dependencies), mỗi đỉnh biểu thị một module, một cạnh nối có hướng chỉ sự phụ thuộc 
của module này vào module kia. Một ví dụ về đồ thị biểu diễn sự phụ thuộc của các modules
trong một ứng dụng web: 
\begin{figure}[H] % places figure environment here   
    \centering % Centers Graphic
    \includegraphics[width=0.6\textwidth]{assets/web_grp.png} 
    \caption{Ví dụ đồ thị sự phụ thuộc của các modules
    trong một ứng dụng web} % Creates caption underneath graph
    \label{fig:gr_1.2.1}
\end{figure}

\subsubsection{Mạng giao thông}
Mô hình đồ thị được sử dụng trong nhiều loại mạng giao thông khác nhau như đường bộ, 
hàng không, mạng đường sắt và mạng chuyển phát. \\

\textit{Định tuyến trong hàng không} \\
Mô hình mạng hàng không có thể được biểu diễn bằng đồ thị với mỗi sân bay là một đỉnh.
Các chuyến bay từ sân bay này tới sân bay khác có thể được biểu diễn bằng một cạnh 
có hướng từ sân bay cất cánh (đỉnh bắt đầu) đến sân bay hạ cánh (đỉnh kết thúc). \\

\textit{Mạng đường bộ} \\
Mô hình đồ thị có thế được sử dụng để biểu diễn mạng đường bộ mà trong đó các đỉnh 
thể hiện các giao lộ và các cạnh thể hiện đường đi. Khi tất cả các cong đường trong mạng 
đều là đường 2 chiều thì ta có thể biểu diễn mạng bằng một đơn đồ thị vô hướng. Tuy 
nhiên trong thực tế ta thường gặp trong mạng giao thông thì một số đường là 2 chiều
và một số khác là 1 chiều. Để biểu diễn mạng này ta dùng cạnh vô hướng để biểu diễn 
đường 2 chiều và cạnh có hướng để biểu diễn đường một chiều.

\subsubsection{Mạng sinh học}
Nhiều khía cạnh của khoa học sinh học có thể được mô hình hóa bằng cách sử dụng đồ thị.\\
\textit{Đồ thị chồng chéo ngách trong hệ sinh thái} \\
Đồ thị được sử dụng trong nhiều mô hình liên quan đến sự tương tác của các loài động vật khác nhau. \\
Ví dụ, sự cạnh tranh giữa các loài trong hệ sinh thái có thể được mô hình hóa bằng cách sử dụng đồ thị chồng chéo thích hợp. Mỗi loài được biểu diễn bằng một đỉnh. Một cạnh vô hướng nối hai đỉnh nếu hai loài đại diện bởi các đỉnh này cạnh tranh (nghĩa là một số nguồn thức ăn mà chúng sử dụng là giống nhau). 
Đồ thị chồng chéo ngách là một đồ thị đơn giản vì không cần vòng lặp hoặc nhiều cạnh trong mô hình này. Biểu đồ trong Hình 11 mô hình hệ sinh thái của một khu rừng. Từ biểu đồ này, chúng ta thấy rằng sóc và gấu trúc cạnh tranh nhưng quạ và chuột thì không.
\begin{figure}[H] % places figure environment here   
    \centering % Centers Graphic
    \includegraphics[width=0.5\textwidth]{assets/dothi_chongcheongach.png} 
    \caption{Đồ thị chồng chéo ngách} % Creates caption underneath graph
\end{figure}

\textit{Đồ thị tương tác protein} \\
Tương tác protein trong tế bào sống xảy ra khi hai hoặc nhiều protein trong tế bào đó liên kết với nhau để thực hiện một chức năng sinh học. Bởi vì các tương tác protein là quan trọng đối với hầu hết các chức năng sinh học, nhiều nhà khoa học đang nghiên cứu để phát hiện ra các protein mới và các tương tác cơ bản giữa các protein. Tương tác protein trong tế bào có thể được mô hình hóa bằng cách sử dụng đồ thị tương tác protein,
một đồ thị vô hướng trong đó mỗi protein được biểu diễn bằng một đỉnh, với một cạnh nối các đỉnh biểu thị từng cặp protein tương tác với nhau. Việc xác định tương tác protein thực sự trong tế bào là một vấn đề khó khăn, vì các thí nghiệm thường tạo ra kết quả dương tính giả, kết luận rằng hai protein tương tác khi chúng thực sự không tương tác. Đồ thị tương tác protein có thể được sử dụng để suy ra thông tin sinh học quan trọng, chẳng hạn như bằng cách xác định các protein quan trọng nhất cho các chức năng khác nhau và chức năng của các protein mới được phát hiện.
\begin{figure}[H] % places figure environment here   
    \centering % Centers Graphic
    \includegraphics[width=0.5\textwidth]{assets/dothi_tuongtacprotein.png} 
    \caption{Đồ thị tương tác protein} % Creates caption underneath graph
\end{figure}
Bởi vì có hàng ngàn loại protein khác nhau trong một tế bào điển hình, nên đồ thị tương tác protein của một tế bào là cực kỳ lớn và phức tạp. Ví dụ, tế bào nấm men có hơn 6.000 protein, và hơn 80.000 tương tác giữa chúng đã được biết đến, và tế bào người có hơn 100.000 protein, có lẽ khoảng 1.000.000 tương tác giữa chúng. Các đỉnh và cạnh bổ sung được thêm vào đồ thị tương tác protein khi các protein mới và tương tác giữa các protein được phát hiện. Do sự phức tạp của đồ thị tương tác protein, chúng thường được chia thành các đồ thị nhỏ hơn được gọi là mô-đun đại diện cho các nhóm protein có liên quan đến một chức năng cụ thể của tế bào. Hình trên minh họa một mô-đun của đồ thị tương tác protein được mô tả trong [Bo04], bao gồm phức hợp của các protein phân loại RNA trong tế bào người. Để tìm hiểu thêm về đồ thị tương tác protein, hãy xem [Bo04], [Ne10] và [Hu07].

\subsubsection{Mạng ngữ nghĩa}
Các mô hình đồ thị được sử dụng rộng rãi trong việc hiểu ngôn ngữ tự nhiên và trong việc truy xuất thông tin. Hiểu ngôn ngữ tự nhiên (NLU) là chủ đề của máy móc điều khiển để tháo rời và phân tích cú pháp lời nói của con người. Mục tiêu của nó là cho phép máy móc hiểu và giao tiếp như con người. Truy xuất thông tin (IR) là chủ đề thu thập thông tin từ tập hợp các nguồn dựa trên nhiều loại tìm kiếm khác nhau. Sự hiểu biết ngôn ngữ tự nhiên là công nghệ cho phép khi chúng tôi trò chuyện với các nhân viên dịch vụ khách hàng tự động. Những tiến bộ trong NLU được thể hiện rõ khi giao tiếp giữa con người và máy móc liên tục được cải thiện. Khi chúng tôi thực hiện tìm kiếm trên web, chúng tôi tận dụng lợi thế của nhiều tiến bộ trong việc truy xuất thông tin được thực hiện trong những thập kỷ gần đây.\\
Trong các mô hình đồ thị cho các ứng dụng NLU và IR, các đỉnh thường đại diện cho các từ, cụm từ hoặc câu, và các cạnh thể hiện các kết nối liên quan đến ý nghĩa của các đối tượng này.\\
Trong mạng ngữ nghĩa, các đỉnh được sử dụng để biểu diễn các từ và các cạnh vô hướng được sử dụng để kết nối các đỉnh khi một quan hệ ngữ nghĩa giữ giữa các từ này. Quan hệ ngữ nghĩa là quan hệ giữa hai hoặc nhiều từ dựa trên nghĩa của từ. Ví dụ, chúng ta có thể xây dựng một đồ thị trong đó các đỉnh đại diện cho danh từ và hai đỉnh được nối với nhau khi chúng có ý nghĩa tương tự. Ví dụ, tên của các quốc gia khác nhau có ý nghĩa tương tự, tên của các loại rau khác nhau cũng vậy. Để xác định danh từ nào có nghĩa tương tự, có thể kiểm tra phần lớn văn bản. Các danh từ trong văn bản được phân tách bằng dấu liên kết (chẳng hạn như “hoặc” hoặc “và”) hoặc dấu phẩy, hoặc xuất hiện trong danh sách, được cho là có nghĩa tương tự.\\
Ví dụ, xem xét các sách về nông nghiệp, chúng ta có thể xác định rằng các danh từ đại diện cho tên của các loại trái cây như bơ, bưởi, ổi, xoài, đu đủ và mãng cầu xiêm cũng có nghĩa tương tự. Các nhà nghiên cứu thực hiện phương pháp này bằng cách sử dụng British National Corpus, một tập hợp các văn bản tiếng Anh với 100.000.000 từ, tạo ra một biểu đồ với gần 100.000 đỉnh, đại diện cho danh từ và 500.000 liên kết, kết nối các đỉnh đại diện cho các cặp từ có nghĩa tương tự\\
\begin{figure}[H] % places figure environment here   
    \centering % Centers Graphic
    \includegraphics[width=0.8\textwidth]{assets/dothi_mangngunghia.png} 
    \caption{Đồ thị mạng ngữ nghĩa} % Creates caption underneath graph
\end{figure}
Hình trên hiển thị một đồ thị nhỏ trong đó các đỉnh đại diện cho danh từ và các cạnh nối các từ có nghĩa tương tự. Biểu đồ này tập trung xung quanh từ chuột. Biểu đồ minh họa rằng có hai ý nghĩa riêng biệt đối với chuột. Nó có thể đề cập đến một con vật hoặc nó có thể đề cập đến phần cứng máy tính. Khi một chương trình NLU gặp từ mouse trong một câu, nó có thể xem những từ nào có nghĩa tương tự sẽ phù hợp với câu để giúp xác định xem con chuột ám chỉ động vật hay phần cứng máy tính trong câu đó.\\

\subsubsection{Mạng thi đấu}
\textit{Thi đấu vòng tròn} \\
Một giải đấu mà mỗi đội đấu với các đội khác đúng một lần và không cho phép hòa được gọi là giải đấu vòng tròn một lượt. Các giải đấu như vậy có thể được mô hình hóa bằng cách sử dụng đồ thị có hướng trong đó mỗi đội được biểu diễn bằng một đỉnh. Lưu ý rằng (a, b) là một cạnh nếu đội a thắng đội b. Biểu đồ này là một biểu đồ có hướng đơn giản, không chứa vòng lặp hoặc nhiều cạnh có hướng (vì không có hai đội chơi với nhau nhiều hơn một lần). Mô hình đồ thị có hướng như vậy được trình bày trong hình. Chúng tôi thấy rằng Đội 1 là đội bất bại trong giải đấu này,
và Đội 3 bất phân thắng bại.\\
\begin{figure}[H] % places figure environment here   
    \centering % Centers Graphic
    \includegraphics[width=0.6\textwidth]{assets/dothi_thidauvongtron.png} 
    \caption{Thi đấu vòng tròn} % Creates caption underneath graph
\end{figure}
\textit{Thi đấu loại trực tiếp} \\
Một giải đấu mà mỗi thí sinh bị loại sau một lần thua được gọi là giải đấu loại trực tiếp. Các giải đấu loại trực tiếp thường được sử dụng trong thể thao, bao gồm giải vô địch quần vợt và giải vô địch bóng rổ NCAA hàng năm. Chúng ta có thể lập mô hình một giải đấu như vậy bằng cách sử dụng một đỉnh để đại diện cho mỗi trò chơi và một cạnh có hướng để kết nối trò chơi với trò chơi tiếp theo mà người chiến thắng trong trò chơi này đã chơi. Biểu đồ trong hình dưới thể hiện các trò chơi của 16 đội cuối cùng trong năm 2010 NCAA của
giải đấu bóng rổ.
\begin{figure}[H] % places figure environment here   
    \centering % Centers Graphic
    \includegraphics[width=0.7\textwidth]{assets/dothi_thidauloaitructiep.png} 
    \caption{Thi đấu loại trực tiếp} % Creates caption underneath graph
\end{figure}
\subsection{Mô hình và ứng dụng của đồ thị trong thực tế}



\section{Bài 2}

\subsection{Chia để trị}

\subsubsection{Ý tưởng}
Phương pháp chia để trị dựa trên 2 thao tác chính:
\begin{itemize}
    \item Chia (\textit{devide}): phân rã bài toán ban đầu thành các bài toán con có kích thước
    nhỏ hơn, có cùng cách giải.
    \item Trị (\textit{conque}): giải từng bài toán con (theo cách tương tự bài toán đầu - đệ
    qui) rồi tổng hợp các lời giải để nhận kết quả của bài toán ban đầu.
\end{itemize}

Việc “Phân rã”: thực hiện trên miền dữ liệu (chia miền dữ liệu thành các miền
nhỏ hơn tương đương 1 bài toán con)

\subsubsection{Mô hình và lược đồ}
Xét bài toàn $P$ trên miền dữ liệu $R$.

Gọi $D\_C(R)$ là thuật giải $P$ trên miền dữ liệu $R$.

Nếu $R$ có thể phân rã thành $n$ miền con: $R = R_1 \cup R_2 \cup ... \cup R_n$

Với $R_0$ là miền đủ nhỏ để  $D\_C(R)$ có lời giải, ta có lược đồ giải thuật chia để trị như sau:

\begin{lstlisting}[style=algo]
    Divide_Conque($R$):
        if($R = R_0$):
            solve Divide_Conque($R_0$)
        else
            divide $R$ to $R_1, R_2, ..., R_n$
            for ($i = 1, 2, ..., n$):
                Divide_Conque($R_i$)
            Combine and get result
    end
\end{lstlisting}

\subsubsection{Phân tích và đánh giá}
Để phân tích và đánh giá độ phức tạp của thuật toán, ta thực hiện 2 công đoạn

\begin{itemize}
    \item Xây dựng công thức truy hồi đánh giá độ phức tạp thuật toán
    \item Giải công thức truy hồi xác định độ phức tạp thuật toán.
        \begin{itemize}
            \item Phép thế liên tiếp
            \item Sử dụng định lí chính
        \end{itemize}
\end{itemize}


\subsubsection{Ví dụ}
Ta xét bài toán \textit{tìm kiếm nhị phân trên một mảng được sắp xếp}.

Cho dãy $n$ phần tử được sắp theo thứ tự (\textit{tăng dần}) và một giá trị $x$ 
bất kỳ. Kiểm tra xem phần tử  $x$ có trong dãy không?






This homework answers the problem set sequentially. 

\begin{enumerate}

\item {\it Download the US Presidential Elections data set {\tt uspresidentialelections.dta} from the course ILIAS site. Load the data set in {\tt R}}. % <--- For future Homework sets you of course have to change the questions.

\begin{verbatim}
Copy your R Code to answer the question here.
\end{verbatim}

\item {\it Describe the dataset. What variables does it contain? How many observations are there? What time span does it cover?}

Please type your answer here.

\begin{verbatim}
Put the right R command here.
\end{verbatim}

\newpage % The \newpage command starts a new blank page. 
\item {\it Compute measures of central tendency and variability of the variables {\tt vote} and {\tt growth} using {\tt R}. Use the numerical measures of central tendency and variability discussed in class. Describe them in your own words and make a nice table. Plot the distribution of both variables using a boxplot and histogram. Make sure to make your plots as nice-looking as possible. Especially, include a title and label the axes.}

Your answer goes here

\begin{verbatim}
3 R commands here
\end{verbatim}

Potentially, your answer continues here.

\begin{verbatim}
4 R commands here
\end{verbatim}

And more of your answer here.

\begin{verbatim}
And more space for your R commands.
\end{verbatim}

\begin{verbatim}
# This is the code to produce the first boxplot.
pdf(file = "box1.pdf")
boxplot(us_data$vote, horizontal = T,
        main  = "A Boxplot of the Variable Vote",
        names = "Vote",
        xlab  = "Range of Votes")
dev.off()
\end{verbatim}

% Now we also want to include the graph in our write up.
 \begin{figure}[H] % places figure environment here   
    \centering % Centers Graphic
    \includegraphics[width=0.9\textwidth]{box1.pdf} 
    \caption{Boxplot of Incumbent Vote share} % Creates caption underneath graph
  \end{figure}


% Now all we need to answer the question is a neat table. The easiest way to get a nice looking table is to browse to http://www.tablesgenerator.com. Generate your table just like in Word or any other WYSIWYG editor. Then copy and include the code here. I already did that for you.

\begin{table}[H]
\centering
\begin{tabular}{llllllll}
\multicolumn{1}{c}{\textbf{Variable}} & \multicolumn{1}{c}{\textit{Mean}} & \multicolumn{1}{c}{\textit{Median}} & \textit{Mode} & \textit{Var} & \textit{SD} & \textit{Range} & \textit{IQR} \\ \hline
Vote                                  & x                                 & x                                   & x             & x            & x           & x              & x            \\
Growth                                & x                                 & x                                   & x             & x            & x           & x              & x           
\end{tabular}
\caption{Measures of central tendency and variability.}
\label{my-label}
\end{table}
% You just have to exchange the x for the right value.

\newpage
\item {\it Make a bar plot of the party affiliation of incumbent presidential candidates.}

Include the code for the bar plot and the plot here. % Don't forget to use the verbatim environment for the code.

\newpage
\item  {\it During the presidential campaign in 1992, Bill Clinton's campaign coined the phrase ``It's the economy, stupid!" Let's investigate the relationship between the economy and electoral success. Generate a nice-looking scatterplot of economic growth and vote share. Label the data points with the year of the election. Describe the pattern that you see in your own words.}

Include the code for the scatterplot as well as the plot here.

Then, describe the pattern you see. In the scatterplot we can see that... 

\end{enumerate}

\clearpage
\section*{R-Code}

\begin{verbatim}
Finally, copy and paste the entire script here.

\end{verbatim}

\end{document}
