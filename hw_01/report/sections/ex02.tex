\section{Bài 2}

\subsection{Chia để trị}

\subsubsection{Ý tưởng}
Phương pháp chia để trị dựa trên 2 thao tác chính:
\begin{itemize}
    \item Chia (\textit{devide}): phân rã bài toán ban đầu thành các bài toán con có kích thước
    nhỏ hơn, có cùng cách giải.
    \item Trị (\textit{conque}): giải từng bài toán con (theo cách tương tự bài toán đầu - đệ
    qui) rồi tổng hợp các lời giải để nhận kết quả của bài toán ban đầu.
\end{itemize}

Việc “Phân rã”: thực hiện trên miền dữ liệu (chia miền dữ liệu thành các miền
nhỏ hơn tương đương 1 bài toán con)

\subsubsection{Mô hình và lược đồ}
Xét bài toàn $P$ trên miền dữ liệu $R$.

Gọi $D\_C(R)$ là thuật giải $P$ trên miền dữ liệu $R$.

Nếu $R$ có thể phân rã thành $n$ miền con: $R = R_1 \cup R_2 \cup ... \cup R_n$

Với $R_0$ là miền đủ nhỏ để  $D\_C(R)$ có lời giải, ta có lược đồ giải thuật chia để trị như sau:

\begin{lstlisting}[style=algo]
    Divide_Conque($R$):
        if($R = R_0$):
            solve Divide_Conque($R_0$)
        else
            divide $R$ to $R_1, R_2, ..., R_n$
            for ($i = 1, 2, ..., n$):
                Divide_Conque($R_i$)
            Combine and get result
    end
\end{lstlisting}

\subsubsection{Phân tích và đánh giá}
Để phân tích và đánh giá độ phức tạp của thuật toán, ta thực hiện 2 công đoạn

\begin{itemize}
    \item Xây dựng công thức truy hồi đánh giá độ phức tạp thuật toán
    \item Giải công thức truy hồi xác định độ phức tạp thuật toán.
        \begin{itemize}
            \item Phép thế liên tiếp
            \item Sử dụng định lí chính
        \end{itemize}
\end{itemize}


\subsubsection{Ví dụ}
Ta xét bài toán \textit{tìm kiếm nhị phân trên một mảng được sắp xếp}.

Cho dãy $n$ phần tử được sắp theo thứ tự (\textit{tăng dần}) và một giá trị $x$ 
bất kỳ. Kiểm tra xem phần tử  $x$ có trong dãy không?
