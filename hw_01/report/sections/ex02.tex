\section{Bài 2}

\subsection{Chia để trị}

\subsubsection{Ý tưởng}
Phương pháp chia để trị dựa trên 2 thao tác chính:
\begin{itemize}
    \item Chia (\textit{devide}): phân rã bài toán ban đầu thành các bài toán con có kích thước
    nhỏ hơn, có cùng cách giải.
    \item Trị (\textit{conque}): giải từng bài toán con (theo cách tương tự bài toán đầu - đệ
    qui) rồi tổng hợp các lời giải để nhận kết quả của bài toán ban đầu.
\end{itemize}

Việc “Phân rã”: thực hiện trên miền dữ liệu (chia miền dữ liệu thành các miền
nhỏ hơn tương đương 1 bài toán con)

\subsubsection{Mô hình}
Xét bài toàn $P$ trên miền dữ liệu $R$.

Gọi $D\_C(R)$ là thuật giải $P$ trên miền dữ liệu $R$.

Nếu $R$ có thể phân rã thành $n$ miền con: $R = R_1 \cup R_2 \cup ... \cup R_n$

Với $R_0$ là miền đủ nhỏ để  $D\_C(R)$ có lời giải, ta có lược đồ giải thuật chia để trị như sau:

\begin{lstlisting}[style=algo]
    Divide_Conque($R$):
        if($R = R_0$)
\end{lstlisting}