\section{Bài 2}

\subsection{Chia để trị}

\subsubsection{Ý tưởng}
Phương pháp chia để trị dựa trên 2 thao tác chính:
\begin{itemize}
    \item Chia (\textit{devide}): phân rã bài toán ban đầu thành các bài toán con có kích thước
    nhỏ hơn, có cùng cách giải.
    \item Trị (\textit{conque}): giải từng bài toán con (theo cách tương tự bài toán đầu - đệ
    qui) rồi tổng hợp các lời giải để nhận kết quả của bài toán ban đầu.
\end{itemize}

Việc “Phân rã”: thực hiện trên miền dữ liệu (chia miền dữ liệu thành các miền
nhỏ hơn tương đương 1 bài toán con)

\subsubsection{Mô hình và lược đồ}
Xét bài toàn $P$ trên miền dữ liệu $R$.

Gọi $D\_C(R)$ là thuật giải $P$ trên miền dữ liệu $R$.

Nếu $R$ có thể phân rã thành $n$ miền con: $R = R_1 \cup R_2 \cup ... \cup R_n$

Với $R_0$ là miền đủ nhỏ để  $D\_C(R)$ có lời giải, ta có lược đồ giải thuật chia để trị như sau:

\begin{lstlisting}[style=algo]
    Divide_Conque($R$):
        if($R = R_0$):
            solve Divide_Conque($R_0$)
        else
            divide $R$ to $R_1, R_2, ..., R_n$
            for ($i = 1, 2, ..., n$):
                Divide_Conque($R_i$)
            Combine and get result
    end
\end{lstlisting}

\subsubsection{Phân tích và đánh giá}
Để phân tích và đánh giá độ phức tạp của thuật toán, ta thực hiện 2 công đoạn

\begin{itemize}
    \item Xây dựng công thức truy hồi đánh giá độ phức tạp thuật toán
    \item Giải công thức truy hồi xác định độ phức tạp thuật toán.
        \begin{itemize}
            \item Phép thế liên tiếp
            \item Sử dụng định lí chính
        \end{itemize}
\end{itemize}


\subsubsection{Ví dụ}
Ta xét bài toán \textit{tìm kiếm nhị phân trên một mảng được sắp xếp}.

\begin{itemize}
    \item Cho dãy $n$ phần tử được sắp theo thứ tự (\textit{tăng dần}) và một giá trị $x$ 
    bất kỳ. Kiểm tra xem phần tử  $x$ có trong dãy không?

    \item Phân tích ý tưởng: so sánh giá trị $x$ với phần tử giữa của dãy tìm kiếm. Dựa
    vào giá trị này sẽ quyết định giới hạn tìm kiếm ở bước kế tiếp là nửa trước
    hay nửa sau dãy.

    \item Lược đồ của thuật toán như sau:

\end{itemize}

\begin{lstlisting}[style=algo]
    BinarySearch($a, x, L, R$):
        // Search element $x$ in array $a$ from position $L$ to $R$
        if ($L = R$):
            return ($x = a_L$ ? $L$ : $-1$)
        else
            $M = (L + R)/2$
            if ($x = a_M$) 
                return ($M$)
            else
                if ($x < a_M$)
                    BinarySearch($a, x, L, R$)
                else 
                    BinarySearch($a, x, M + 1, R$)
                endif
            endif
        endif
    end
\end{lstlisting}

\textit{Tính đúng của thuật toán}

Ta chứng mình bằng quy nạp như sau 
\begin{itemize}
    \item Cơ sở quy nạp: $n = R - L + 1 = 1$ (dãy có 1 phần tử)
    \begin{itemize}
        \item Câu lệnh \lstinline{return (x = a_L ? L : -1)} trả về giá trị $L$ hoặc $-1$
    \end{itemize}
    
    \item Giả thiết quy nạp: Thuật toán đúng với mọi dãy có độ dài $n = R-L+1$.
    Hay hàm \lstinline{BinarySearch(a, x, L, R)} trả về đúng kết quả tìm kiếm $x$ với mọi dãy
    có đội dài $1 \leq n' \leq n = R - L + 1$

    \item Tổng quát: Chứng minh thuật toán đúng với $n+1 = R-L+2$
    \begin{itemize}
        \item Đặt $M=(L+R+1)/2$, ta có $L \leq M \leq R$
        \item Nếu $x = a_M$ thì kết quả trả về là $M$: đúng
        \item Nếu $x < a_M$ thì kết quả là của bài toán tìm $x$ trong tập $a_L, ... , a_M$.
        Theo giả thiết quy nạp thì \lstinline{BinarySearch(a, x, L, R)} đúng vì 
        $1 \leq M - L + 1 = (R - L + 1)/2 + 1 \leq R - L + 1$
        \item Tương tự với $x > a_M$
    \end{itemize}
\end{itemize}

\textit{Độ phức tạp của thuật toán}

\begin{equation*}
    T(n) = 
    \begin{cases}
        1 & \text{when } n = 1 \\
        T(n/2) + 1 & \text{when } n > 1
    \end{cases}
\end{equation*}

Do đó $T(n) = O(\log n)$

\textit{Source code: } \lstinline{https://github.com/batman0911/dma_homework/blob/master/hw_01/src/main.ipynb}


\subsection{Quay lui}

\subsubsection{Ý tưởng}
Ý tưởng của phương pháp Quay lui (Back tracking): Theo nguyên tắc vét cạn (xét qua tất cả các trường hợp có thể xảy ra để tìm kết quả), nhưng chỉ xét những trường hợp “khả quan”.

Tại mỗi bước, nếu có một lựa chọn được chấp thuận thì ghi nhận lại lựa chọn này và tiến hành các bước thử tiếp theo. Còn ngược lại không có lựa chọn nào thích hợp thì quay lại bước trước.

Phương pháp quay lui được sử dụng để giải bài toán liệt kê các cấu hình:
\begin{itemize}
    \item Mỗi cấu hình được xây dựng bằng cách xác định từng phần tử
    \item Mỗi phần tử được chọn bằng cách thử các khả năng \textit{\textbf{có thể}}
    \item Độ dài cấu hình tùy thuộc bài toán
        \begin{itemize}
            \item Xác định trước: sinh dãy độ dài n
            \item Không xác định trước: đường đi
        \end{itemize}
\end{itemize}

\subsubsection{Mô hình}
Không gian nghiệm của bài toán (tập khả năng) $X=\{(x_1,x_2,…,x_n)\}$ gồm các cấu hình liệt kê có dạng ($x_1,x_2,…,x_n$) cần được xây dựng:
\begin{itemize}
    \item Cho $x_1$ nhận lần lượt các giá trị có thể. Với mỗi giá trị thử gán cho $x_1$ thì:
    \item Cho $x_2$ nhận lần lượt các giá trị có thể. Với mỗi giá trị thử gán cho $x_2$ thì xét khả năng chọn $x_3,…, x_n$, => cấu hình tìm được ($x_1, x_2, ……, x_n$).
    \item \textbf{Tổng quát:} Tại mỗi bước i: Xây dựng thành phần $x_i$
    \begin{itemize}
        \item Xác định $x_i$ theo khả năng v.
        \item Nếu gặp điều kiện dừng (i = n hoặc $x_i$ thỏa mãn điều kiện dừng) thì ta có được một lời giải, ngược lại thì tiến hành bước i+1 để xác định $x_{i+1}$
        \item Nếu không có một khả năng nào chấp nhận được cho $x_i$ thì lại lùi lại bước trước để xác định lại thành phần $x_{i-1}$ – sự quay lui
    \end{itemize}
\end{itemize}

\subsubsection{Lược đồ}
Lược đồ phương pháp quay lui như sau:
\begin{itemize}
    \item [] Try(i) $\equiv \textit{//Sinh thành phần thứ i của cấu hình}$
        \begin{itemize}
            \item [] for (v thuộc tập khả năng thành phần nghiệm $x_i$)
                \begin{itemize}
                    \item [] \textbf{if} ($x_i$ chấp nhận được giá trị v)
                        \begin{itemize}
                            \item [] $x_i$ = v;
                            \item [] <Ghi nhận trạng thái chấp nhận v>;
                            \item [] \textbf{if} (gặp điều kiện dừng) \textit{//i=n hoặc $x_i$ thỏa mãn điều kiện dừng}
                            \item [] \qquad <xử-lí-nghiệm>;
                            \item [] $\textbf{else} \qquad \textit{//lời gọi sinh thành phần tiếp theo của cấu hình}$
                                \item [] \qquad Try (i + 1)
                            \item [] <Khôi phục trạng thái chưa chấp nhận v>;
                        \end{itemize}
                    \item [] \textbf{endif;}
                \end{itemize}
            \item [] \quad \textbf{endfor;}
        \end{itemize}
    \item [] \qquad \textbf{End.} \hfill \textbf{Lời gọi ban đầu:} Try(1)
\end{itemize}


\subsubsection{Phân tích và đánh giá}
Ta thực hiện 2 công đoạn phân tích và đánh giá thuật toán như sau:

\begin{itemize}
    \item Công thức truy hồi
        \begin{itemize}
            \item [] $
                    T(n) = \left\{\begin{array}{lcr}
                    1 & \text{khi} & n \leq 1\\
                    dT(n-1)+1 & \text{khi} & n > 1
                    \end{array} \right.
                   $
            \item [] d: max (hoặc giá trị trung bình) lực lượng của tập khả năng của các thành phần nghiệm $x_i$
        \end{itemize}
    
    \item Giải công thức truy hồi: $T(n) = O(d\sp{n})$
        \begin{itemize}
            \item Là trường hợp xấu nhất (vét cạn)
            \item Trong triển khai thuật toán thời gian thực tế có thể giảm xuống
            \item [] \qquad for (v thuộc tập khả năng thành phần nghiệm $x_i$)
            \item [] \qquad \qquad if ($x_i$ chấp nhận được giá trị v)
        \end{itemize}
\end{itemize}

\textbf{Tính đúng của thuật toán:}
Sự hiển nhiên trong tính đúng của thuật toán quay lui:
\begin{itemize}
    \item Mỗi phần tử trong cấu hình nghiệm được chọn trong tập khả năng
        \begin{itemize}
            \item [] for (v thuộc tập khả năng thành phần nghiệm $x_i$)
            \item [] \hspace{1cm} if ( $x_i$ chấp nhận được giá trị v)
            \item [] \hspace{1.5cm} $x_i$ = v;
        \end{itemize}
        
    \item Tính dừng của lời gọi đệ quy và cho ra kết quả đúng
    \item [] \hspace{1.8cm} \textbf{if} (gặp điều kiện dừng) \textit{//i=n hoặc xi thỏa mãn đk dừng}
    \item [] \hspace{2.5cm} <xử-lí-nghiệm>;
    \item [] \hspace{1.8cm} \textbf{else} \textit{//lời gọi sinh thành phần tiếp theo của cấu hình}
    \item [] \hspace{2.5cm} Try (i + 1)
    
    \item \hspace{0.5cm} Đệ quy tiến Try(1)/$x_1$ $\rightarrow$ Try(2)/$x_2$ $\rightarrow$ … $\rightarrow$ Try(m)/$x_m$ 
        \begin{itemize}
            \item Đủ thành phần nghiệm m=n (n - hữu hạn)
            \item Đến thành phần thỏa mãn điều kiện tức là $x_m$ thỏa mãn điều kiện nghiệm
        \end{itemize}
\end{itemize}


\subsubsection{Ví dụ}
Ta xét bài toán \textit{Liệt kê dãy k\_ary có độ dài n}
\begin{itemize}
    \item Phân tích
        \begin{itemize}
            \item Input: n, k
            \item Output: Các dãy X = ($x_1, x_2, …,x_n$) trong đó $x_i$ =0,1,..k-1
            \item Dùng giải thuật Try(i) để sinh giá trị $x_i$
            \item Nếu i=n thì in giá trị nghiệm X, ngược lại sinh tiếp $x_{i+1}$ bằng Try(i+1)
        \end{itemize}
    \item Lược đồ giải thuật:
        \begin{itemize}
            \item [] Try(i) $\equiv$
                \begin{itemize}
                    \item [] \textbf{for} (v=0..k-1) \textit{//v nhận giá trị từ 0 đến k-1}
                        \begin{itemize}
                            \item [] $x_i$ = v ;
                            \item [] \textbf{if} (i=n) printResult (X) ;
                            \item [] \textbf{else} Try(i+1) ;
                        \end{itemize}
                    \item [] \textbf{endfor};
                \end{itemize}
            \item [] \enskip \textbf{End.} \hfill Lời gọi ban đầu \textbf{Try(1)}
        \end{itemize}
\end{itemize}

\subsection{Quy hoạch động}

\subsubsection{Ý tưởng}
\begin{itemize}
    \item Quy hoạch động (Dynamic programing) là phương pháp giải các bài toán bằng cách kết hợp lời giải của các bài toán con theo nguyên tắc:
        \begin{itemize}
            \item Giải tất cả các bài toán con (một lần)
            \item Lưu lời giải của các bài toán vào một bảng
            \item Phối hợp các bài toán con để nhận lời giải bài toán ban đầu
        \end{itemize}
    \item Cách phát biểu khác: Một bài toán giải bằng quy hoạch động được phân rã thành các bài toán con và bài toán lớn sẽ được giải quyết thông qua các bài toán con này (bằng các phép truy hồi)
    \item Phương pháp quy hoạch động thường được dùng cho các bài toán tìm giá trị (hoặc giá trị tối ưu)
\end{itemize}

\subsubsection{Các bước giải bài toán}
\begin{enumerate}
    \item{Nhận dạng bài toán}
    \item[]Các bài toán có các đặc trưng sau thì có thể giải bằng QHĐ
        \begin{itemize}
            \item Bài toán có thể giải bằng đệ qui và tìm lời giải tối ưu.
            \item Bài toán có thể phân rã thành nhiều bài toán con mà sự phối hợp lời giải của các bài toán con sẽ cho ta lời giải của bài toán ban đầu.
            \item Quá trình tìm ra lời giải của bài toán ban đầu từ các bài toán con đơn giản hơn được thực hiện qua một số hữu hạn các bước có tính truy hồi.
        \end{itemize}
    \item Xây dựng công thức truy hồi
        \begin{itemize}
            \item Đưa bài toán về 1 dạng cơ bản, và triển khai ý tưởng của dạng bài toán đó để nhanh chóng nhận ra hướng thiết lập công thức truy hồi.
            \item Đây là bước khó nhất và cũng quan trọng nhất trong toàn bộ quá trình thiết kế thuật toán cho bài toán.
        \end{itemize}
    \item Xác định và xây dựng cơ sở QHĐ
        \begin{itemize}
            \item Dựa vào công thức truy hồi để nhận ra các bài toán cơ sở.
            \item Dựa vào ý nghĩa của công thức truy hồi để thiết lập giá trị cho cơ sở.
        \end{itemize}
    \item Dựng bảng phương án
        \begin{itemize}
            \item Dựa vào công thức truy hồi để tính giá trị các ô trong bảng phương án.
            \item Chú ý: bảng phương án có thể 1 chiều, 2 chiều hoặc nhiểu hơn
        \end{itemize}
    \item Tìm kết quả tối ưu
        \begin{itemize}
            \item Xác định vị trí chứa kết quả tối ưu của bài toán trên bảng phương án.
            \item Chú ý: ngoài kết quả tối ưu, ô chứa kết quả tối ưu còn là điểm bắt đầu cho quá trình truy vết tìm nghiệm => lưu tọa độ của ô đó.
        \end{itemize}
    \item Truy vết liệt kê thành phần nghiệm
        \begin{itemize}
            \item Từ điểm bắt đầu là vị trí chứa kết quả tối ưu
            \item Truy ngược lại về điểm bắt đầu của nghiệm: có thể là những ô đầu tiên trong bảng phương án (bài toán cơ sở), có thể là ô của bảng phương án đạt giá trị đầu.
        \end{itemize}
\end{enumerate}

\subsubsection{Ví dụ}
\begin{enumerate}
    \item Bài toán "Dãy con đơn điệu tăng dài nhất"
    \begin{description}
        \item[Bài toán:] Tìm dãy con dài nhất của một dãy đã cho. Các phần tử có thể không liên tiếp.
        \item[Phân tích:]
            \begin{itemize}
                \item[]
                \item Input: (a,n)
                \item Output: số lớn nhất các phần tử của dãy theo thứ tự tăng dần
                \item Hàm tối ưu L(i): Độ dài dãy con đơn điệu tăng dài nhất đến phần tử i
                \item[]Là độ dài các dãy con dài nhất đến j cộng 1 khi ghép thêm ai vào sau, với
                điều kiện j<i, aj<ai
                \item Công thức truy hồi: L(i) = max{L(j)}+1 với j<i, aj<ai
                \item Cơ sở của thuật toán: L(0) = 0; L(1) = 1
            \end{itemize}
            \begin{lstlisting}[style=algo]
def lis(_list):
    longest = [_list[0]]
    current = [_list[0]]
    for i in _list[1:]:
        if i >= current[-1]:
            current.append(i)
        else:
            if len(longest) < len(current):
                longest = current
            current = [i]
    if len(longest) < len(current):
            longest = current
    return longest
            \end{lstlisting}
    \end{description}


    \item Bài toán "Xếp balo 0-1"
    \begin{description}
        \item[Bài toán:]Có N đồ vật với trọng lượng và giá trị tương ứng(wi,vi). Tìm cách cho các vật vào balo có trọng lượng W sao cho đạt giá trị cao
        nhất. Mỗi vật chỉ được chọn 1 lần
        \item[Phân tích:]
            \begin{itemize}
                \item[]
                \item Input: n đồ vật, (wi,vi) i=1..n, Túi có trọng lượng tối đa W
                \item Output: V=tổng giá trị lớn nhất của các đồ vật vào balo
                \item Hàm tối ưu dp[i,j]: Giá trị lớn nhất khi chọn đồ vật từ 1 tới i với trọng lượng balo j (i=0...n,j=0...W)
                \item[]Nếu không chọn đồ vật thứ i thì: dp[i,j]=dp[i-1,j]
                \item[]Nếu chọn đồ vật thứ i thì: dp[i,j]=dp[i-1,j-wi]+vi (Điều kiện wi<=j)
                \item Công thức truy hồi: 
                \item[]dp[i,j]=Max(dp[i-1,j], dp[i-1,j-wi]+vi)
                \item Cơ sở quy hoạch động
                \item[]dp[i,0]=0
                \item[]dp[0,j]=0
            \end{itemize}
        
        \begin{lstlisting}[style=algo]
def solve(weights, values, capacity):
    items = list(zip(weights, values))
    def dp(i, cp):
        if i == len(items):
            return 0.0
        w, v = items[i]
        ans = dp(i + 1, cp)
        if cp >= w:
            ans = max(ans, dp(i + 1, cp - w) + v)
        return ans
    return int(dp(0, capacity))
        \end{lstlisting}
    \end{description}
\end{enumerate}  