\section{Bài 3}
Trong bài này chúng ta sẽ xem xét bài toán dóng hàng
toàn cục 2 chuỗi DNA sử dụng phương pháp quy hoạch động 
với giải thuật Needleman-Wunsch.

\subsection{Khái niệm}
Có nhiều cách để phát biểu bài toán dóng hàng trình tự 
(cho DNA, RNA hoặc 2 chuỗi kí tự). Hãy cùng điểm qua vài khái niệm.

\begin{itemize}
    \item Dóng hàng trình tự: là một cách sắp xếp trình tự của DNA,
    RNA hoặc protein để xác định các vùng giống nhau có thể là
    hệ quả của mối quan hệ chức năng, cấu trúc hoặc tiến hóa giữa
    các trình tự.
    \item Một cách dóng hàng của 2 chuỗi được thiết lập bằng cách thêm
    các khoảng trống (dấu cách) vào các vị trí bất kì trên 2 chuỗi này
    để chúng có cùng độ dài và không có 2 khoảng trống nào có cùng vị trí 
    trên 2 chuỗi
    \item Chuỗi con chung dài nhất (longest common subsequence, LCS):
    là chuỗi trình tự chứa nhiều kí tự giống nhau nhất của hai hay
    nhiều chuỗi.
\end{itemize}
Ví dụ một cách dóng hàng với 2 chuỗi S (\lstinline{interestingly}) 
và T (\lstinline{bioinformatics}) như sau
\begin{center}
    \lstinline{-i--nterestingly} \\
    \lstinline{bioinformatics--}
\end{center}

Một cách tổng quát, chúng ta có thể chấm điểm cho mỗi cặp kí tự được dóng hàng.
Gọi $\Sigma$ là tập các kí tự và "-" là kí tự đặc biệt kí hiệu cho dấu cách. 
Sự tương tự của các cặp kí tự trong 2 chuỗi có thể được biểu diễn thông qua 
một ma trận $\delta$ mà $\delta(x,y)$ là điểm của cặp $x$ và $y$ với 
$x, y \in \Sigma \cup \{-\}$. \\
Bài toán dóng hàng toàn cục có thể mô hình hóa bằng cách tìm một các dóng hàng A
nào đó để cực đại $\sum_{\{x, y \} \in A} \delta(x,y)$. Cách dóng hàng này được
gọi là cách tối ưu (\textit{optimal alignment}). \\
Khi một cặp kí tự được sắp xếp và giống nhau, liên hệ giữa chúng
 được gọi là \textit{match}, ngược
lại gọi là \textit{mismatch}. Khi dấu cách được thêm vào chuỗi thứ nhất mối liên hệ 
là \textit{insert}, khi được thêm vào chuỗi thứ 2 thì được gọi là \textit{delete}. \\
Trong ví dụ trên chúng ta có 5 \textit{matches}, 6 \textit{mismatchs}, 3 \textit{inserts}
và 2 \textit{deletes}.

\subsection{Giải thuật Needleman-Wunsch}
Xét 2 chuỗi kí tự $S[1...n]$ và $T[1...m]$. Để tìm các dóng hàng toàn cục tối ưu cho 
bài toán này chúng ta có thể nghĩ tới giải thuật vét cạn bằng cách sinh ra mọi cách 
dóng hàng và tìm xem cách nào có điểm cao nhất. Tuy nhiên cách tiếp cận này có thời
gian tính toán theo hàm mũ. Trong phần này, chúng ta sẽ xem xét một giải thuật hiệu quả
áp dụng \textit{quy hoạch đông} có tên là \textit{Needleman-Wunsch}. \\
Chúng ta thiết kế một hàm đệ quy $V(i,j)$ với 2 trường hợp: (1) $i = 0$ hoặc $j = 0$
và (2) cả $i > 0$ và $j > 0$. \\
Với trường hợp (1) khi $i = 0$ hoặc $j = 0$ chúng ta dóng hàng chuỗi bằng 1 chuỗi rỗng.
Hay nói cách khác là ta chỉ cần thêm hoặc xóa kí tự. Chúng ta có các phương trình:
\begin{equation}
    \begin{aligned}
        V(0, 0) &= 0 & \\
        V(0, j) &= V(0, j - 1) + \delta(-, T[j]) && \text{thêm j lần} \\
        V(i, 0) &= V(i -1, 0) + \delta(S[i], -) && \text{xóa i lần} 
    \end{aligned}
\end{equation}

