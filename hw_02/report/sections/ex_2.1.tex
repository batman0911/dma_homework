\subsection{Thuật toán Dijkstra tìm đường đi ngắn nhất trên đồ thị}
Trước khi giải bài toán tổng quát, ta hãy xem xét một ví dụ đơn giản sau đây:

\begin{figure}[H] % places figure environment here   
    \centering % Centers Graphic
    \includegraphics[width=0.4\textwidth]{assets/gr_01.png} 
    \caption{Ví dụ đơn đồ thị với trọng số } % Creates caption underneath graph
    \label{fig:gr_01}
\end{figure}

Tìm đường đi ngắn nhất từ đỉnh $a$ đến đỉnh $z$ trong đồ thị ở hình \ref{fig:gr_01}.\\ 

Trước hết, bắt đầu từ đỉnh $a$, có 2 đỉnh kết nối với $a$ là $b$ và $d$ với trọng số 
lần lượt là $4$ và $2$, đương nhiên $d$ là đỉnh gần $a$ nhất. Chúng ta có thể tìm đỉnh 
gần nhất thứ 2 bằng cách liệt kê tất cả các đường đi với $d$ là đỉnh bắt đầu và đỉnh 
kết thúc không nằm trong tập $\{a, d\}$. Tính từ đỉnh $d$, chỉ có một đường khả dĩ là $e$.
Từ đó ta có tập $\{a, d, e\}$. Tương tự như vậy ta tìm đỉnh tiếp theo với cạnh có trọng số 
nhỏ nhất với $e$ là đỉnh bắt đầu và đỉnh kết thúc không thuộc tập $\{a, d, e\}$, ta được 
đỉnh $z$. Đường đi cuối cùng là $a \to d \to e \to z$. \\

Ví dụ trên thể hiện nguyên lý tổng quát của thuật toán Dijkstra đó là đường đi từ đỉnh $a$
tói đỉnh $z$ có thể được xây dựng bằng cách liệt kê các đường đi và tìm đường có trọng số 
nhỏ nhất để thêm đỉnh đó vào tập đường đi.

Một cách tổng quát ta có thể trình bày thuật toán Dijkstra với lược đồ sau:

\begin{tcolorbox}[colframe=gray,colback=gray!20,left=10pt,right=10pt,top=10pt,bottom=10pt]
    \textbf{Thuật toán Dijkstra} \\
    $G$ là một đơn đồ thị kết nối với trọng số mỗi cạnh dương \\
    $G$ có các đỉnh $a = v_0, v_1, ..., v_n = z$, trọng số 
    $w(v_i, v_j)$ với $w(v_i, v_j) = \infty$ nếu $\{v_i, v_j\}$ không thuộc 
    tập cạnh của $G$

    \begin{itemize}
        \item [] for i = 1 to n 
        \begin{itemize}
            \item [] $L(v_i) = \infty$
        \end{itemize} 
        \item [] $L(a) = 0$
        \item [] $S = \emptyset $ (Khởi tạo các nhãn với nhãn của $a$ là $0$, 
        các nhãn khác bằng $\infty$ và tập $S$ rỗng)
        \item [] while $z \notin S$
        \begin{itemize}
          \item [] $u = a$ đỉnh không thuộc $S$ với $L(u)$ nhỏ nhất
          \item [] $S = S \cup \{u\}$
          \item [] for $v \notin S$
          \begin{itemize}
            \item [] if $L(u) + w(u,v) < L(v)$ then $L(v) = L(u) + w(u,v)$
            (Bước này thêm đỉnh gần nhất vào tập $S$ và cập nhật lại hàm chi phí)
          \end{itemize}
        \end{itemize}
        return $L(z)$ ($L(z)$ là độ dài của đường đi ngắn nhất từ $a$ tới $z$)
    \end{itemize}
  \end{tcolorbox}

\textit{Nhận xét: } Thuật toán Dijkstra đem lại
sự tường minh mà không phải quét qua toàn bộ cấu hình của bài toán, 
ta chỉ cập nhật đỉnh gần với đỉnh cuối cùng nhất, điều đó làm giảm không gian lưu 
trữ so với cách vét cạn (khi ta phải lưu toàn bộ đường đi và chi phí của mọi đường
đi khả dĩ từ đỉnh $a$ tới đỉnh $z$). Độ phức tạp của thuật toán là $O(n^2)$. 
Ta có thể dễ dàng thu được vì ban đầu khi chọn 1 đỉnh, ta cần $(n-1)$ phép so sánh
để tìm được đỉnh gần nhất đầu tiên, tiếp theo ta cần $(n-2)$ phép tính để tìm đỉnh 
gần nhất thứ 2, tiếp tục như vậy cho đến khi thu được đỉnh đích. Số phép tính là tổng 
từ $1$ đến $(n-1)$. \\

Trong thực tế lập trình, chúng ta không nhất thiết phải lưu giá trị cạnh bằng một 
số lớn (vô cùng) mà có thể đánh bằng $0$ và kiểm tra thêm một điều kiện trong vòng 
lặp để tránh sự cồng kềnh của ma trận kề. \\

\href{https://github.com/batman0911/dma_homework/blob/master/hw_02/src/shortest_path.ipynb}{python code} 
    cho bài toán.