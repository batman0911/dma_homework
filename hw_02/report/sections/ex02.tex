\section{Bài 2}
Trong phần này chúng ta sẽ xem xét bài toán "Travelling salesman problem" (bài toán 
người giao hàng) - là một trong những bài toán kinh điển và nổi tiếng trong lớp các
bài toán về đồ thị nói chung và bài toán tìm đường đi ngắn nhất nói riêng. \\

\textit{Bài toán: Cho một danh sách các thành phố và khoảng cách từng cặp một, 
tìm đường đi ngắn nhất đi qua mỗi thành phố đúng một lần và quay về điểm xuất phát} \\


Chúng ta có thể mô hình bài toán trên bằng đồ thị như sau: \textit{Cho một đơn đồ thị đầy đủ, 
với các cạnh có trọng số, tìm đường đi xuất phát từ một đỉnh đi qua tất cả các đỉnh 
của đồ thị mỗi đỉnh đúng một lần với tổng trọng số (chi phí) nhỏ nhất.}

Trước hết chúng ta giải bài toán con đơn giản hơn là tìm đường đi với chi phí nhỏ nhất từ
một đỉnh $a$ tới đỉnh $z$ cho trước. Trong bài toán này, không nhất thiết ta phải đi qua 
tất cả các đỉnh của đồ thị. Có nhiều cách tiếp cận bài toán, cách đầu tiên mà ta có thể nghĩ 
ngay tới là \textit{vét cạn (brute force)}, tuy nhiên việc liệt kê toàn bộ cấu hình \
(toàn bộ các đường đi khả dĩ từ $a$ đến $z$) của bài toán rất tốn thời gian. Sau đây ta cùng
xem xét một thuật toán hiệu quả hơn là thuật toán Dijkstra.

\subsection{Thuật toán Dijkstra tìm đường đi ngắn nhất trên đồ thị}
Trước khi giải bài toán tổng quát, ta hãy xem xét một ví dụ đơn giản sau đây:

\begin{figure}[H] % places figure environment here   
    \centering % Centers Graphic
    \includegraphics[width=0.4\textwidth]{assets/gr_01.png} 
    \caption{Ví dụ đơn đồ thị với trọng số } % Creates caption underneath graph
    \label{fig:gr_01}
\end{figure}

Tìm đường đi ngắn nhất từ đỉnh $a$ đến đỉnh $z$ trong đồ thị ở hình \ref{fig:gr_01}.\\
Trước hết, bắt đầu từ đỉnh $a$, có 2 đỉnh kết nối với $a$ là $b$ và $d$ với trọng số 
lần lượt là $4$ và $2$, đương nhiên $d$ là đỉnh gần $a$ nhất. Chúng ta có thể tìm đỉnh 
gần nhất thứ 2 bằng cách liệt kê tất cả các đường đi với $d$ là đỉnh bắt đầu và đỉnh 
kết thúc không nằm trong tập $\{a, d\}$. Tính từ đỉnh $d$, chỉ có một đường khả dĩ là $e$.
Từ đó ta có tập $\{a, d, e\}$. Tương tự như vậy ta tìm đỉnh tiếp theo với cạnh có trọng số 
nhỏ nhất với $e$ là đỉnh bắt đầu và đỉnh kết thúc không thuộc tập $\{a, d, e\}$, ta được 
đỉnh $z$. Đường đi cuối cùng là $a \to d \to e \to z$.

Ví dụ trên thể hiện nguyên lý tổng quát của thuật toán Dijkstra đó là đường đi từ đỉnh $a$
tói đỉnh $z$ có thể được xây dựng bằng cách liệt kê các đường đi và tìm đường có trọng số 
nhỏ nhất để thêm đỉnh đó vào tập đường đi.

Một cách tổng quát ta có thể trình bày thuật toán Dijkstra với lược đồ sau:
