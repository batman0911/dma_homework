
\subsection{Bài toán người đưa hàng (TSP)}
\subsubsection{Thuật toán tìm nghiệm chính xác}
Trong mục này chúng ta sẽ xem xét bài toán người đưa hàng đã đề cập ở đầu bài.
Sau khi giải bài toán tìm đường đi ngắn nhất ở mục trước, ta biết rằng bài toán
người đưa hàng không gì khác ngoài việc tìm được đi ngắn nhất khi ta đặt điểm
kết thúc chính là điểm bắt đầu. 

\href{https://github.com/batman0911/dma_homework/blob/master/hw_02/src/travelling_salesman.ipynb}{python code} 
    cho bài toán. \\

\textit{Nhận xét: } Như đã nói ở phần trước, thuật toán này không làm giảm độ phức 
tạp tính toán so với cách vét cạn và là $O (n!)$.
Độ phức tạp này tăng rất nhanh theo $n$ . Ví dụ khi bắt đầu từ 1 đỉnh, ta có $(n-1)!$
chu trình Hamilton, tới đỉnh thứ 2 ta còn $(n-2)!$ chu trình Hamilton và tiếp tục như vậy.
Bởi vì ta có thể đi theo thứ tự ngược lại trong chu trình Hamilton nên số chu trình cần
sinh ra để có được lời giải là $(n-1)!/2$. Với số đỉnh bằng $25$, số chu trình được sinh 
ra là $24!/2 \sim 3.1 \times 10^{23}$ - một con số lớn khủng khiếp. Trong thực tế, 
một đơn vị vận chuyển như \textit{Giao hàng tiết kiệm} giao hàng triệu đơn hàng mỗi ngày 
tới các địa điểm khác nhau! Do đó ta gần như không thế tìm được lời giải chính xác của bài 
toán TSP trong thực tế mà cần phải sử dụng các phương pháp xấp xỉ để tìm được một 
đường đi \textit{đủ tốt}.

\subsubsection{Phương pháp xấp xỉ}

